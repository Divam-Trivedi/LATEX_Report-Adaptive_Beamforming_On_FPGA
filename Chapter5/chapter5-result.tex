\chapter{Results \& Discussions}
\indent\indent Here, we describe and display the outputs of the entire flow as well as explain how the final output came out to be. First, we'll be discussing the outputs from VHDL of thhe various modules implemented as mentioned before.

\section{Simulation outputs of individual modules}
All the modules used in the SMI algorithm are individually implemented whose outputs are as shown.
The following outputs have been simulated in Vivado software provided by Xilinx in the VHDL language.

\subsection{Output of Floating Point Units}
\textbf{Adder}
Addition of two 32-bit floating point numbers is implemented as shown below in the IEEE-754 format

\begin{figure}[h]
\centering
\includegraphics[scale=0.35]{Chapter5/Figures/adder}		
\caption{ \label{fig:1}Floating Point Addition}
\end{figure} 

\textbf{Multiplier}
Multiplication of two 32-bit floating point numbers is implemented as shown below in the IEEE-754 format

\begin{figure}[h]
\centering
\includegraphics[scale=0.35]{Chapter5/Figures/multiplier}		
\caption{ \label{fig:1}Floating Point Multiplication}
\end{figure} 

\textbf{Division}
Division of two 32-bit floating point numbers is implemented as shown below in the IEEE-754 format

\begin{figure}[h]
\centering
\includegraphics[scale=0.35]{Chapter5/Figures/division}		
\caption{ \label{fig:1}Floating Point Division}
\end{figure} 

\subsection{Matrix Multiplier}
Two NxN matrix having complex floating point values are taken and the result of the matrix multiplication is shown here. Two outputs are received, one real part of each element and one output for the imaginary part of each element.

\begin{figure}[h]
\centering
\includegraphics[scale=0.30]{Chapter5/Figures/mat_mult}		
\caption{ \label{fig:1} Floating point matrix multiplication}
\end{figure} 

\subsection{Matrix Inversion}
An NxN matrix having complex floating point values is taken and the result of the inversion of the matrix is shown here. Two outputs are received, one real part of each element and one output for the imaginary part of each element.

Matrix inversion of a 2 x 2 matrix
\begin{figure}[h]
\centering
\includegraphics[scale=0.3]{Chapter5/Figures/mat_inv2x2}		
\caption{ \label{fig:1} Floating point matrix inversion of 2x2 matrix}
\end{figure}

Matrix inversion of a 3 x 3 matrix
\begin{figure}[h]
\centering
\includegraphics[scale=0.30]{Chapter5/Figures/mat_inv3x3}		
\caption{ \label{fig:1} Floating point matrix inversion of 3x3 matrix}
\end{figure}

\subsection{Element wise matrix multiplication}
Two NxN matrix having complex floating point values is taken and the result ofthe element wise multiplication of the matrices is shown here. Two outputs are received, one real part of each element and one output for the imaginary part of each element.

\begin{figure}[h]
\centering
\includegraphics[scale=0.3]{Chapter5/Figures/ele_mat}		
\caption{ \label{fig:1} Floating point Elemenet wise matrix multiplication}
\end{figure}

\subsection{Top module - SMI block design}
Here we show how 


\begin{figure}[h]
\centering
\includegraphics[scale=0.35]{Chapter5/Figures/final_module}		
\caption{ \label{fig:1} Final output as required}
\end{figure}
